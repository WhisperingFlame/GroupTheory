\documentclass[UTF8]{book}

\usepackage{CTEX}
\usepackage{geometry}
\geometry{left=3cm, right=3cm, top=4cm, bottom=4cm}


\begin{document}

\let\cleardoublepage\clearpage
\begin{center}
\chapter{前言:对对称性的追求}
\end{center}

对称的事物在自然界中是如此的奇特,以至于我们的祖先一定很早就注意到了它们。的确,人们对对称的结构给予了神秘的特殊地位。
古希腊人对几何形状的痴迷让他们找出了“柏拉图立体”(正多面体)。他们也往往用对称的纹样来装饰他们的建筑。在古代世界,
对称与完美具有相同的含义。毕竟,再没有比圆或者球更圆满的形状了。人们认为太阳与行星都绕着地球做圆周运动,直到很久以后才发现他们的
轨道实际上是不那么完美的椭圆。

当然,自然中的大部分物体在形状上并没有对称性,或者只有很低的对称性,但也有很多是接近对称的。橙子的形状就接近于一个理想的球体。
人体的左右两侧也几乎是对称的,但并不完全(译者注:手性),古人应当也注意到了这一点。不过,我们并不知道他们是否将这种在对称性上的缺失视作一种
人需要去弥补的缺陷。(译者:其实八成是的,中世纪的炼金术士追求的或许某种程度上带有这种含义。)

尽管人们很早就意识到具有高度对称性的事物是特殊的,(一个有趣的事实是)关于对称性如何产生的数学结构直到19世纪才被系统性地研究。
这并不意味着对称的图案并不为人所知或是被人们忽略了。西班牙的摩尔人(Moors)在建造他们的宫殿时使用了足足17种方式平铺平面,
说明他们对于对称图式的高度了解。


埃瓦里斯特·伽罗华(Évariste Galois)在研究四次多项式的根时,发现这一问题意味着寻找构成一种特殊数学结构的代换(注:原文substitution)的集合,
我们后来称之为群(注:Group)。在物理学中,对晶体的研究引出了极为规则的图案以及对他们的对称性的描述。在二十世纪,随着量子力学的发展,
对称性在对自然的研究中占据了中心地位。

基础粒子物理中标准模型的提出进一步增强了对称性的重要性。这一模型说明自然界在微观层面展现出比宏观层面更多的对称性。在宇宙学上,
这意味着大爆炸产生的宇宙具有高度对称的结构,尽管它的大部分对称性在如今已经不是那么显而易见了。就像一块远古的陶片一样,它的某些部分没能在亿万年后留存下来。
而仅仅只残存了一些碎片。这个有趣的想法与古希腊的关于完美的理念不谋而合。(注:大致就是,完美的时代逐渐劣化到了当下)。
在大爆炸产生了我们的宇宙之后,它是具有完美的对称性的吗?那么,它的对称性是在宇宙的演化中逐步破裂的吗(注:用的是shattered,什么艾尔登法环)。
抑或是宇宙本身就带有着内在的缺陷,而导致了最终对称性的破缺呢?如今,一些物理学家在尝试使用一些关于绝对对称的宇宙的理论模型来回答这个意义深远的问题,例如超弦理论。

自然界中有些对称性是如此的稀松平常,以至于它们甚至可能难以被我们意识到。一个本科生在同一个实验中的结果不会随着实验所进行的时间和位置而改变。
即这些实验的结果不会受到时间和空间变动的影响。换言之,它们对时间和空间具有平移不变性。不过,还存在着更加细微的对称现象。伟大的伽利略·伽利雷
(Galileo·Galilei)做了一个“平凡”的观察:当你的船在稳定的风力下航行于平静的海面时,你可以闭上眼睛然后“感觉”不到自己在移动。
更棒的是,你在此时进行的实验的结果会和站定不动时一样!在当今,你可以在航班在巡航高度飞行时把手中的酒杯放在一边,而不用担心它会洒出来。
伽利略以伟大的天才将这一观察提升为相对性原理:在静止或者匀速运动状态下,物理定律都是相同的!然而,如果速度发生变化,你就会感觉到这一点
(一个小小的湍流就会把你的酒洒出来)。我们所经验的日常世界看上去是复杂的(注:这里应该是从对称性的角度,指“不那么对称”),是因为我们的世界是由摩擦力
支配的。在摩擦力的影响可以被忽略的情况下,简单性和对称性(在某种程度上这两个概念是相似的)就会表现出来。

根据量子力学,物理是在希尔伯特空间中展开的。尽管可能有点奇怪,但随着它不断地在实验上被验证,我们已经接受了这个观点。当然,
用希尔伯特空间中的矢量来标识物理系统的这样一种抽象最终会被发现是不完全的。但在目前,我们还无法想象这种更加完全的描述方式,
那将会包含某种诡异的新数学结构。数学家们发明的数学结构与自然界如此相通是一个具有深远意义的谜团,它暗示了我们大脑中连接的方式。
不论其根源是什么,在希尔伯特空间中表示自然界事物的这些数学结构引起了极大的物理兴趣。其中尤为突出的就是\textit{群}。群由特定的公理所规定,
描述了在希尔伯特空间中的变换。

由于物理学家主要关注群在希尔伯特空间中是如何运作的,本书主要关注群的表示(注:representations)。我们会以\textit{注释}的形式
在书中穿插涉及的数学概念。我们也会简化证明过程,读者可以在许多杰出的教材中找到这些内容。从群表示出发,我们会关注它们如何相乘,并展示如何为
可能遇到的物理应用建立群不变量(注:group invariants)。我们也会讨论群所包含的子群的表示的嵌入(注:embeddings)。本书将会包含大量的表格。

本书从研究\textit{有限}群开始。如同名字所示,这些群包含有限个对称操作。最小的有限群只有两个元素,不过它们的元素数量并没有上限:
$n$个字符的置换操作构成了一个包含$n!$个元素的有限群。在物理中,有限群有着许多的应用。其中大部分是在晶体学或是新材料的性质上。
在基本粒子物理学中,只有那些较小的有限群被应用;但在有着更多维度的对象以及三类神秘的基本粒子中,情况则并不相同。特别地,散在群(注:sporadic groups),
包含26个有限群的特殊集合,目前还只在数学上被研究,等待着人们发现它们的应用。

我们随后会考虑\textit{连续}的对称变换,例如任意角度的旋转,或是无限制的(注:open-ended)时间平移。连续的变换可以被视作从生成元(注:generators)产生的无限小步的重复。
一般来说,这些生成元构成称为李代数(注:Lie algebras)的代数结构。我们会从最简单的连续群以及它们的李代数开始,在它们的基础上建立起更复杂的案例。
我们将按照邓金(Dynkin)的方式展开李代数,使用邓金符号与邓金图(注:Dynkin notations and diagrams)。我们将格外注意特殊群和它们的表示。特别地,我们会探讨魔方中的群论。
我们还会建立连续群和有限群之间的关联,毕竟大部分有限群可以被归结为连续群的子群。

本书也会探讨一些非紧的(注:non-compact)对称性,时空对称性的表示。例如庞加莱(注:Poincaré)群和共形群(注:conformal groups)。
标准模型和大统一理论的群论方面也会涉及。五种特殊李群的代数结构将在本书中细致地探讨。我们还会介绍两种广义李代数,分别是超李代数(super-Lie algebra)及其分类,
以及无限维仿射的卡茨-穆迪代数(注:Kac-Moody algebra)。

我为普林斯顿高等研究院(注:Institute of Advanced Study)和阿斯彭物理中心(注:Aspen Center for Physics)的款待致以诚挚的谢意。
本书的一些精彩部分是在这些地方写就的。感谢L. Brink教授和J. Patera教授,D.Belyaev博士、Sung-Soo Kim博士还有C. Luhn博士对本书手稿的批判性阅读,
以及他们提出的许多宝贵建议。最后,我非常感激我的妻子Lillian。是她对我的的耐心、鼓励以及理解让这本书得以完成。

\end{document}

